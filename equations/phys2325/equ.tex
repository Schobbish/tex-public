\documentclass{article}
\usepackage{textcomp}
\usepackage{amsmath}
\usepackage{physics}

\numberwithin{equation}{section}

\title{PHYS 2325 Equations}
\author{}
\date{}
% I have been informed that "symbols used as subscripts and superscripts are
% roman if descriptive" but they can be "italic if they represent quantities
% or variables" (including x and y for direction) (source: nist.gov).

\begin{document}
    \maketitle
    \tableofcontents

    \newpage
    \section{Kinematics}
        \subsection*{Some Constant Acceleration Equations}
            Surprisingly enough, these equations apply when acceleration is constant, such as when you drop something.
            \begin{equation}
                v = v_0 + at
            \end{equation}
            \begin{equation}
                \Delta x = v_0t + \frac{1}{2}at^2
            \end{equation}
            \begin{equation}
                v^2 = v_0^2 + 2a\Delta x
            \end{equation}
            \begin{equation}
                \Delta x = \frac{1}{2}(v_0 + v)t
            \end{equation}

        \subsection*{3D motion}
            You should also be able to find displacement, velocity, and acceleration in 3D using vectors, e.g.

            \begin{equation}
                \Delta \vec{r} = \vec{r}_2 - \vec{r}_1 =
                (x_2 - x_1)\hat{i} + (y_2 - y_1)\hat{j} + (z_2 - z_1)\hat{k}
            \end{equation}
            \begin{equation}
                \vec{v}_{\mathrm{ave}} = \frac{\Delta\vec{r}}{\Delta t}
            \end{equation}
            \begin{equation}
                \vec{v} = \dv{r}{t} =
                \dv{x}{t}\hat{i} + \dv{y}{t}\hat{j} + \dv{z}{t}\hat{k}
            \end{equation}
            And similarly with $\vec{a}$.

        \subsection*{Circular motion}
            Particle goes in circle.
            \begin{equation}
                a_{\mathrm{radial}} = \frac{v^2}{r}
            \end{equation}
            $a_{\mathrm{radial}}$ refers to the component of acceleration which is perpendicular to velocity (i.e. pointing inwards) and gives the change in direction. This applies to uniform circular motion and non-uniform circular motion.
            \begin{equation}
                a_{\mathrm{tan}} = \dv{|\vec{v}|}{t}
            \end{equation}
            $a_{\mathrm{tan}}$ refers to the component of acceleration which is parallel to velocity and gives the change in speed. In uniform circular motion, this value is zero.
            \begin{equation}
                T = \frac{2\pi r}{v}
            \end{equation}
            Just a reminder that $T$, the period, is the time it takes for the particle to complete one revolution of the circle. It is the reciprocal of frequency.

        \subsection*{Projectile motion}
            These equations are typically used in 2D motion where one component has constant acceleration and the other component has zero acceleration and there is some initial velocity. They describe a parabolic trajectory with equation:
            \begin{equation}
                y(x) = x\tan(\theta) - \frac{ax^2}{2v_0^2\cos^2(\theta_0)}
            \end{equation}
            Note that in these equations it is assumed that the $y$-component of motion has the constant acceleration because of gravity.

            I don't think these equations are that helpful since they just do the component finding for you but here they are anyway:
            \begin{equation}
                x(t) = (v_0cos(\theta))t
            \end{equation}
            \begin{equation}
                y(t) = (v_0sin(\theta))t - \frac{1}{2}at^2
            \end{equation}

            Max height of projectile motion. This will not be given in the exam.
            \begin{equation}
                h = \frac{v_0^2\sin^2 \theta_0}{2a}
            \end{equation}

            Max range (how far it goes horizontally). This will also not be given in the exam.
            \begin{equation}
                x_\mathrm{max} = \frac{v_0^2}{a}\sin(2\theta_0)
            \end{equation}
            Shows that the range is maximized when the launch angle $\theta_0 = 45\text{\textdegree}$.
\end{document}
