\documentclass{article}[11pt]
\usepackage{amsmath}

\numberwithin{equation}{section}

\title{AP Physics 1 Equations}
\author{}
\date{}

\begin{document}
    \maketitle
    \tableofcontents

    \newpage
    \section{Kinematics}
        \subsection*{Variables in this Section}
            In this section, a variable with subscript $f$ indicates the final
            value, and a variable with subscript $i$ indicates the initial
            value. The exam does not use the $f$ subscript (i.e., there is no
            subscript on a final quantity), and instead of an $i$ subscript it
            uses a $0$ (zero/naught) subscript.

            The arrow above a variable indicates a vector quantity. The exam
            does not use this notation. Instead, it uses a $x$ or a $y$
            subscript to indicate if the object is moving from side to side or
            up and down.

            \begin{description}
                \item[$\vec{a}$] is acceleration. Given as $a_x$ or $a_y$ on
                the exam depending on the dimension of travel.
                \item[$\vec{a}_g$] is acceleration due to gravity at Earth's
                surface, and is equal to $9.8\, \frac{\text{m}}{\text{s}^2}$.
                On the exam this is given as $g$.
                \item[$\Delta \vec{d}$] is displacement. The exam uses $x$ and
                $y$ variables, so $x - x_0$ will be used instead of $\Delta
                \vec{d}$ on the exam.
                \item[$t$] is time. Thus, $\Delta t$ is change in time and is
                equal to $t_f - t_i$. The $\Delta$ is optional and is not used
                on the exam.
                \item[$\vec{v}$] is velocity. Thus, $\Delta \vec{v}$ is change
                in velocity and is equal to $\vec{v}_f - \vec{v}_i$. Given as
                $v_x$ or $v_y$ on the exam depending on the dimension of
                travel.
            \end{description}
        \subsection*{Equations}
            Definition of acceleration as the rate of change of velocity.
            \emph{Not given on the exam.}
            \begin{equation}
                \vec{a} = \frac{\Delta \vec{v}}{\Delta t}
            \end{equation}

            Velocity equation for an object with constant acceleration.
            Given as $v_x = v_{x0} + a_xt$ on the exam.
            \begin{equation}
                \vec{v}_f = \vec{v}_i + \vec{a}\Delta t
            \end{equation}

            Position equation for an object with constant acceleration.
            Given as $x = x_0 + v_{x0}t + \frac{1}{2}a_xt^2$ on the exam.
            Note that moving $x_0$ to LHS will give $x - x_0$,
            which is equal to $\Delta\vec{d}$.
            \begin{equation}
                \Delta \vec{d} =
                \vec{v}_i\Delta t + \frac{1}{2}\vec{a}\Delta t^2
            \end{equation}

            Equation relating velocity and displacement for
            constant-acceleration motion. Given as
            $v^2_x = v^2_{x0} + 2a_x(x - x_0)$ on the exam.
            \begin{equation}
                \vec{v}^2_f = \vec{v}^2_i + 2\vec{a}\Delta\vec{d}
            \end{equation}
\end{document}
